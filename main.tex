\documentclass[cm-default,no-math,SlantFont,CJKnumber,a4papper,oneside]{ctexbook}
\usepackage[urlcolor = blue,colorlinks = true,citecolor = black,linkcolor = black]{hyperref}
\usepackage{indentfirst}
\usepackage{CJKnumb}
\punctstyle{CCT}

%!!重要!!%这是防止 - -- --- 等特殊符号失去在 LaTeX 下原油的意义。
\defaultfontfeatures{Mapping=tex-text}

\setCJKmainfont[BoldFont=song-bold]{Adobe Song Std}

% Main Font is Song
\setCJKmainfont[Mapping=tex-text,BoldFont=song-bold]{Adobe Song Std}
\setCJKfamilyfont{song}[Mapping=tex-text]{Adobe Song Std}
\setCJKfamilyfont{tt}[Mapping=tex-text]{WenQuanYi Zen Hei Mono}
\setCJKfamilyfont{heiti}[Mapping=tex-text]{Adobe Heiti Std}
\setCJKfamilyfont{kai}[Mapping=tex-text]{AR PL KaitiM GB}

\title{计算机和程序}
\author{\sc Microcai \thanks{KiKi的老公}}

\begin{document}

\maketitle

\tableofcontents

\part{计算机原理}

\chapter{数和二进制}

\chapter{编码的一些概念}
\section{BCD}
\section{ASCII}

\chapter{你倒是算啊!}

\chapter{程序是个什么东西?}

\chapter{计算机的结构}



\part{程序设计基础}

\chapter{数据结构}
\chapter{算法}

\part{C语言基础}

\part{图形的力量} \nopagebreak
\begin{table}[c]
到目前为止,我们的系统都还只有字符界面。但是你知道,一个真正的桌面系统,绝对不可能是只有字符而已。
我们熟悉的图标,菜单,这些东西都是怎么来的呢?如何实现的呢?本部分就带领大家进入图形界面的时代。
\end{table}

\chapter{图形界面由来}

不得不说,图形界面非常吸引人。但是你知道它的历史么?

\chapter{发明图形前的一些故事}

鼠标的发明

Jobs去施乐的一次参观。

\chapter{Apple,Windows 和 X11(UNIX)}
\section{Apple}
Jobs 去施乐回来后,大受刺激,马上命人开发图形界面操作的 Apple. Apple-II 就此诞生。
\section{Windows}
Apple 热卖后,随IBM发家的微软也开始图形界面操作系统的开发。
\section{X11(UNIX)}
UNIX拥有悠久的历史的, UNIX的图形界面和UNIX几乎一样悠久。

\chapter{X11基础}

\chapter{图形加速}

\chapter{来个三维的怎么样?}

\chapter{X11 和 OpenGL }

\part{GTK$+$图形界面编程}

\chapter{序}

\chapter{GTK+的历史}

%\section{Qt}
%为什么要讲Qt呢?我们来个小插曲吧。

%{\begin{quote}\CJKfamily{kai}\input{小故事/KDE-vs-GNOME}\end{quote}}

%//所以,正是因为UNIX世界唯一免费的 Qt 一开始却采用的非GPL授权,才使得 GTK 的诞生显得必要而且迫切。

\section{GIMP: 纪元的启始}

GIMP 原作者 Peter Mattis,Spencer 和 Kimball 这样宣布 GIMP 0.54 版的诞生: 

{\CJKfamily{kai}
GIMP 诞生于让人厌烦的一个课程项目: cs164 (编译器)。那是一个清晨,我们极度缺乏睡眠,正处在极大压力中用 LISP 编写一个编译器,显得非常疲劳。那早已经超过了我们忍耐的极限,但是我们又不得不继续。

当 LISP 无法为 yacc 生成的一个简单语法解释器分配一个所需的17MB内存时,出现了常见的糟糕 core dump。大家都感到心烦,项目最终停止了。我们强烈地感觉到必须写些什么...写点有用的、用 C 语言而不是靠嵌套列表\footnote{LISP语言的基本数据结构}(Nested Lists)来显示位图的东西。这天,GIMP 就诞生了。

它象一只凤凰,从 LISP 和 yacc 的灰烬中光辉地飞出。GIMP 慢慢开始成形...

一个图像处理程序是大家的共识;一个至少能使 在``Windoze''或``Macintoy''下的商业软件不再是唯一选择的程序;一个将提供其它 X 画图和图像工具所没有的特性的程序;一个将帮助保持 \emph{UNIX拥有最完美和自由的应用程序}这个传统的程序。

六个月后,进入早期的 beta 阶段。我们决定此时发布,并开始着手兼容性问题和交叉平台的稳定性,而且觉得程序现在已经能用了,希望有兴趣的程序员加入开发插件和支持不同的文件格式。 
    0.54 版本在 1996 年二月发布,作为第一个正真的专业自由图像处理软件产生了很大影响,并且这也是第一个能够与大型商业图像处理程序竞争的自由软件。
}

    0.54 版本是一个 beta 版本,不过它已经足够稳定,您能在日常工作中使用它。尽管如此,0.54 版本其中一个最大的缺点就是其工具包(滑条,对话框等)是基于一个商业工具包 --- Motif 的。这对像 Linux 一样的系统来说是个大问题,因为如果您想用更快的动态链接的 GIMP,您就不得不买 Motif,而您可能买不起 ------ 许多开发者都是学生,而他们正使用Linux。

当0.60 版本在 1996 年七月发行时,它已经在 S 和 P (Spencer 和 Peter)手下开发了四个月。最主要进步是其工具包,GTK (GIMP Toolkit) 和 GDK (GIMP Drawing Kit),它们解决了对 Motif 的依靠。

GTK 和 GDK 就这样诞生了。

1997 年九月,GTK 和 GDK 从 GIMP 中正式分离,分离后的GTK和GDK被合并,是为GTK+。
%GTK+ 被认为是及其出色的工具包,它被其它开发人员用作编写自己的应用程序。

\section{GNOME: 舍我其谁}

1997年  GTK+ 诞生的时候,就注定了它不平凡的一生。\\
因为这个时候,GNOME正要准备出场了。

\CJKfamily{kai}
MIT 的X Window推出之后就成为UNIX图形界面的标准,但在商业应用上分为两大流派:一派是以Sun公司领导的Open Look阵营,一派是IBM/HP领导的OSF(Open Software Foundation)的Motif,双方经过多年竞争之后,Motif最终获得领先地位。不过,Motif只是一个带有窗口管理器(Window- Manager)的图形界面库(Widget-Library),而非一个真正意义上的GUI界面。经过协商之后IBM/HP与SUN决定将Motif与 Open Look整合,并在此基础上开发出一个名为“CDE(Common Desktop Environment) ”的GUI作为UNIX的标准图形界面。遗憾的是,Motif/CDE和UNIX系统的价格都非常昂贵,而当时微软的Windows发展速度惊人并率先在桌面市场占据垄断地位,而 Unix 界的后起之秀 Linux 也急需一个可靠并且免费的图形界面。


尽管X Window已经非常成熟,也有不少基于X Window的图形界面程序,但它们不是未具备完整的图形操作功能就是价格高昂(如CDE),根本无法用于Linux系统中。如果Linux要获得真正意义上的突破,一套完全免费、功能完善的GUI就非常必要。1996年10月,图形排版工具Lyx的开发者、一位名为Matthias Ettrich的德国人发起了KDE(Kool Desktop Environment)项目,与之前各种基于X Window的图形程序不同的是,KDE并非针对系统管理员,它的用户群被锁定为普通的终端用户,Matthias Ettrich希望KDE能够包含用户日常应用所需要的所有应用程序组件,例如Web浏览器、电子邮件客户端、办公套件、图形图像处理软件等等,将 UNIX/Linux彻底带到桌面。当然,KDE符合GPL规范,以免费和开放源代码的方式运行。


     KDE项目发起后,迅速吸引了一大批高水平的自由软件开发者,这些开发者都希望KDE能够将Linux系统的强大能力与舒适直观的图形界面联结起来,创建最优秀的桌面操作系统。经过艰苦卓绝的共同努力,KDE 1.0终于在1998年的7月12日正式推出。以当时的水平来说,KDE 1.0在技术上可圈可点,它较好的实现了预期的目标,各项功能初步具备,开发人员已经可以很好地使用它了。当然,对用户来说,KDE 1.0远远比不上同时期的Windows 98来得平易近人,KDE 1.0中大量的Bug更是让人头疼。但对开发人员来说,KDE 1.0的推出鼓舞人心,它证明了KDE项目开源协作的开发方式完全可行,开发者对未来充满信心。有必要提到的是,在KDE 1.0版的开发过程中,SUSE、Caldera等Linux商业公司对该项目提供资金上的支持,在1999年,IBM、Corel、RedHat、富士通-西门子等公司也纷纷对KDE项目提供资金和技术支持,自此KDE项目走上了快速发展阶段并长期保持着领先地位。但在2004年之后,GNOME不仅开始在技术上超越前者,也获得更多商业公司的广泛支持,KDE丧失主导地位,其原因就在于KDE选择在Qt平台的基础上开发,而Qt在版权方面的限制让许多商业公司望而却步。

\end{document}